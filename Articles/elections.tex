\article{ELECTIONS}
\section{}
The general elections of the Grand Marshal, the President of the Union, the President of the Undergraduate Council, the elected members of the Sophomore, Junior and Senior Class Councils, the members of the Graduate Council, and all elected members of the Student Senate except those representing the incoming Freshman Class shall be held during the spring semester at a time and place determined by the Student Senate.

\section{}
\begin{enumerate}
\item The conclusion of an election is defined as the time at which its results are released.
\item The results of an election may be released no later than the end of the semester in which the
election is held.
\end{enumerate}

\section{}
\begin{enumerate}
\item The elections of the Freshman Class Council and Freshman members of the Student Senate shall
be held during the fall semester at a time and place determined by the Student Senate.
\item The Officers, Representatives, and Senators for the Freshman Class shall serve for a term
beginning on the date of their election and ending with the conclusion of the next general
election for their respective position.
\end{enumerate}

\section{}
Voting shall be by secret ballot. Only members of the Union at the time of the elections shall have the right to vote or run for office.

\section{Greek Status}
\begin{enumerate}
\item To be eligible for the office of Independent Senator, a student must not be a member of a
fraternity or sorority at Rensselaer.
\item Fraternal affiliation shall be determined by the Dean of Students Office.
\item To be eligible for the office of Greek Senator, a student must be a member of a fraternity or
sorority at Rensselaer. Of the candidates for Greek Senator, the fraternity member with the
highest number of votes and the sorority member with the highest number of votes will be
elected as Greek Senators.
\end{enumerate}

\section{}
\begin{enumerate}
\item The Student Senate shall determine the means whereby a student may have his or her name
placed on the ballot.
\item It shall also be responsible for providing adequate notice of all upcoming elections to the
members of the Union.
\item The Student Senate will be responsible for maintaining proportionate representation of all
members of the Union on the Student Senate, the Graduate Council, the Undergraduate
Council, the Executive Board, and the Judicial Board.
\end{enumerate}

\section{}
All elected offices, unless otherwise specified in this Constitution, shall be for a term beginning upon the date of their election and ending with the conclusion of the next general election for their respective
positions.

\section{}
The candidates for Senators, representatives to the Graduate Council, the Undergraduate Class Councils,
and President of the Undergraduate Council receiving the highest numbers of votes shall be elected.

\section{}
\begin{enumerate}
\item Vacancies in any elective office may be filled by appointment, subject to the provisions of the
constitution or by-laws of the body to which the appointment is made.
\item Such an appointment shall be valid for the remainder of the term of office.
\end{enumerate}

\section{}
\begin{enumerate}
\item Forty-percent of the popular vote shall be necessary for election in the case of Grand Marshal
and the President of the Union.
\item A plurality shall be necessary for election in the case of Undergraduate Class Council officers.
\item The number of representatives to each class council will be specified by its constitution.
\end{enumerate}

\section{}
If no victor can be declared in an election, another election will be held between the tied candidates at a
time and place determined by the Student Senate in its by-laws.

\section{}
\begin{enumerate}
\item The installation of all newly elected persons shall be within one week following the conclusion of
general elections.
\item The President of the Union shall appoint all required members of the Executive Board, excluding
the Graduate Council representative, the Undergraduate Council representative, the Freshman
class member, and the Senate-Executive Board Liaison, within three weeks after the conclusion
of general elections for the President of the Union.
\begin{enumerate}
\item The Undergraduate Council representative shall be appointed by the Undergraduate
Council within three weeks following the conclusion of general elections for the
Undergraduate Class Councils.
\item The Graduate Council representative shall be appointed by the Graduate Council within
three weeks following the conclusion of general elections for the Graduate Council.
\item The Student Senate representative shall be chosen by the Student Senate within three
weeks following the conclusion of general elections for the Student Senate.
\item The President of the Union shall appoint the Freshman class member within three
weeks after the fall elections.
\item The President of the Union shall appoint any discretionary members by the end of the
4
th week of the fall semester.
\item Term of office shall normally be one year but a member will continue to serve until
he/she has been replaced.
\end{enumerate}
\item The Graduate Council shall appoint its at-large members at the first regular meeting following
the conclusion of general elections for the Graduate Council. They shall be selected by the full
eleven member Graduate Council from all applicants. Applications must be solicited for at least
two weeks prior to all selection meetings.
\end{enumerate}

\section{}
To be eligible for election and to hold office, a student must satisfy the eligibility rules established by the
Institute and the additional regulations established by the Student Senate.