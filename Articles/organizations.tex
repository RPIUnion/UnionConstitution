\article{ORGANIZATIONS}
The Union shall be governed by a Student Senate, an Executive Board, a Judicial Board, a Graduate
Council, and an Undergraduate Council.
\begin{enumerate}
\item It shall be constituted under the authority of the Board of Trustees of Rensselaer.
\end{enumerate}


\section{Student Senate}

The Student Senate shall be the chief legislative and policy-making branch of the Union.

\begin{enumerate}
\item The Student Senate shall be presided over by the Grand Marshal.
\item The Student Senate shall consist of 26 voting members: Six Senators from the Graduate Class, four Senators from each undergraduate class, two Independent Senators, and two Greek Senators.
\item It shall be responsible for initiating all legislative action in those areas which affect the Union.
\item It shall pass such by-laws as are necessary for its operation by a $\frac{2}{3}$ vote of its total voting membership.
\item It shall determine the amount of the Union Activity Fee by a $\frac{2}{3}$ vote of its total voting membership.
\item It shall establish procedures for selection and removal of the student members of the committees of the Board of Trustees and all other appointments.
\item It shall be the responsibility of all senators to regularly report to their respective councils.
\end{enumerate}





\section{Executive Board}

The Executive Board shall act as the primary budgeting body of the Union.

\begin{enumerate}
\item The Executive Board shall be presided over by the President of the Union.
\item The Executive Board shall consist of a minimum of 15 voting members, and a maximum of 20
voting members:
\begin{enumerate}
\item The following positions on the Executive Board must be filled: 5 club or intercollegiate athletics representatives, 1 representative each from the Freshman, Sophomore, Junior, Senior, and Graduate classes, a representative of the Graduate Council, a representative of the Undergraduate Council, a Senate-Executive Board Liaison, and 2 Members-at-Large representing any membership of the Union.
\item The following positions on the Executive Board may be filled at the discretion of the President of the Union: 2 club or intercollegiate athletic members, and 3 Members-at-Large.
\item The Senate-Executive Board Liaison, Graduate Council Representative, and Undergraduate Council Representative are appointed by the Senate, Graduate Council, and Undergraduate Council respectively. All other positions are appointed by the President of the Union and approved by a $\frac{2}{3}$ vote of the Senate.
\item If a member of the Executive Board is unable to fulfill his or her duties, a replacement may be appointed by the President of the Union, subject to a majority vote of the Executive Board and a $\frac{2}{3}$ approval of the Senate.
\item The Director of the Union and his or her assistants shall act as advisers to the Executive Board.
\end{enumerate}


\item Such by-laws, as are necessary for its operation, shall be passed by a $\frac{2}{3}$ vote of its total voting
membership followed by a majority vote of the Student Senate.
\item It shall investigate, prepare, and approve the budget for the following fiscal year. The budget
and the records of all additional expenditures and appropriations shall be public documents. 
\item It shall administer the Union budget, appropriate Union funds and manage the business affairs
of any and all facilities operated by the Rensselaer Union.
\item It shall approve the hiring and continuance of all administrative personnel of the Union.
\item It shall decide on the classification of new organizations and changes in the classification of old
ones, as specified in its by-laws.
\end{enumerate}

\section{Judicial Board}
\begin{enumerate}
\item The Judicial Board shall be the primary court of judgment for cases in which the concept of student rights, responsibilities, or conduct is in question.
\begin{enumerate}
\item No other board shall assume any portion of this jurisdiction, unless such a board be comprised solely of students, and the jurisdiction of the board clearly defined.
\item No such board will be constituted nor its jurisdiction changed without the consent of the Student Senate and Institute President.
\end{enumerate}
\item The Judicial Board shall be the sole organization responsible for the interpretation of the Union Constitution.
\item The Judicial Board Chairperson shall be the presiding officer and a non-voting member of the
Judicial Board.
\item The Judicial Board shall consist of 6 voting members and 5 alternate members, all of whom must
be members of the Rensselaer Union.
\begin{enumerate}
\item The chairperson shall appoint new candidates to the Judicial Board subject to a $\frac{2}{3}$ approval of the Judicial Board followed by a $\frac{2}{3}$ approval of the Student Senate.
\item Alternate members are non-voting members of the Judicial Board who may assume voting rights under circumstances specified in the Judicial Board bylaws.
\item All appointments become effective when a vacancy exists in the position to which they are appointed. Appointments shall become effective in the order of their approval by the Judicial Board.
\item Once appointed, members of the Judicial Board remain active until they are no longer a member of the Rensselaer Union, have been removed from office, or have resigned their position.
\end{enumerate}
\item Any charge which specifies a violation of the Union Constitution, the Rensselaer Student Bill of
Rights, or established regulations governing student conduct may be referred to the Judicial
Board. In deciding a case, the Judicial Board may declare any legislation of the Student Senate,
the Executive Board, the Graduate Council, the Undergraduate Council, or the Undergraduate
Class Councils to be unconstitutional.
\item Such by-laws, as are necessary for its operation, shall be passed by a $\frac{2}{3}$ vote of its total voting
membership followed by a majority vote of the Student Senate.

\end{enumerate}

\section{Graduate Council}
\begin{enumerate}
\item The Graduate Council is the graduate legislative branch of the Union. Its voting membership shall consist of the six Graduate Senators and five representatives elected at large. It shall be presided over by the President of the Graduate Council.
\item It shall be responsible for initiating all legislative action within the Union pertaining to the graduate students not specifically dealt with by the Student Senate or the Executive Board.
\item It shall recommend any change in the amount of graduate Class Dues by a majority vote, subject
to a $\frac{2}{3}$ approval of the Student Senate’s total voting membership.
\item The Graduate Council shall be responsible for the expenditure of graduate Class Dues, collected from the members of its class by majority vote.
\item Such by-laws, as are necessary for its operation, shall be passed by a $\frac{2}{3}$ vote of its total voting membership followed by a majority vote of the Student Senate.
\end{enumerate}

\section{Undergraduate Council}
\begin{enumerate}
\item The Undergraduate Council is the undergraduate legislative branch of the Union. It shall consist of the Class President and Vice President of each Undergraduate Class as voting members, and shall be presided over by the President of the Undergraduate Council.
\begin{enumerate}
\item It shall be responsible for coordination and oversight of the four Undergraduate Class Councils.
\item All expenditures of Class Dues by the undergraduate Class Councils are subject to approval by a majority vote of the Undergraduate Council, in accordance with guidelines set forth in the by-laws of the Undergraduate Council.
\item It shall recommend any change in the amount of undergraduate Class Dues by a majority vote, subject to a $\frac{2}{3}$ approval of the Student Senate’s total voting membership.
\item Such by-laws, as are necessary for its operation, shall be passed by a $\frac{2}{3}$ vote of its total voting membership followed by a majority approval of the Student Senate.
\end{enumerate}
\item Each of the Senior, Junior, Sophomore, and Freshman classes shall elect a Class Council. Class Councils exist to promote the interests and unity of their Class.
\begin{enumerate}
\item The Class President is the presiding officer and a non-voting member of the Class
Council. The President may vote to break a tie in any vote of the Class Council.
\item The Class Vice President is an officer and voting member of the Class Council.
\item Voting membership in each Class Council shall include a minimum of eight voting Representative positions, their respective Class Senators, and the Vice President. Additional positions may be defined in the by-laws of the Undergraduate Council.
\item The President and Vice President of each Class Council shall serve as voting members on the Undergraduate Council.
\item Each Class Council shall recommend the expenditure of Class Dues, collected from the members of its class, by majority vote, subject to majority approval of the Undergraduate Council.
\item Each Class Council shall pass legislation commensurate with its duties, subject to regulations established in the bylaws of the Undergraduate Council and this Constitution.
\end{enumerate}
\item No person may hold more than one officer position, as defined by this Constitution or the Undergraduate Council Bylaws, on the same Class Council. The Undergraduate Council President may not hold a voting position nor be an officer of any Class Council.
\end{enumerate}

\section{Parliamentary Authority}
The rules contained in the current edition of Robert's Rules of Order shall govern the subordinate bodies of the Rensselaer Union, as defined in the Rensselaer Union Constitution, in all cases to which they are applicable and in which they are not inconsistent with the Rensselaer Union Constitution and any special rules of order the Union or its subordinate bodies may adopt.