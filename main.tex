\documentclass[12pt]{constitution}
\usepackage{mathpazo}
\title{The Rensselaer Union Constitution}
\author{Effective as of April 13, 2013; as amended April 16, 2015}
\date{}

\setlength{\parindent}{0pt}

\begin{document}
\maketitle
\setcounter{tocdepth}{0}
% \tableofcontents
\newpage

\article{NAME}
The name of this organization shall be the Rensselaer Union, hereinafter referred to as the Union. 

\article{MEMBERSHIP}
All students, both graduate and undergraduate, presently enrolled at Rensselaer Polytechnic Institute, hereinafter referred to as Rensselaer, who have paid the Union Activity Fee, shall constitute the membership of the Union.

\article{PURPOSE}
The purpose of this Union shall be to unite all its members in a commitment to the ideals for which Rensselaer stands, to expand the extracurricular life at Rensselaer, to coordinate all student organizations, to act as a medium through which student opinion may be expressed, and to work with all other members of the Rensselaer community to encourage student initiative and to lead student action in all interests which will serve the welfare and the betterment of Rensselaer.

\article{OFFICERS}
\begin{enumerate}
\item The officers of the Union shall be the Grand Marshal, the President of the Union, the Chairperson of the Judicial Board, the President of the Graduate Council, and the President of the Undergraduate Council.
\item It shall be the duty of the Union officers to meet regularly to discuss policy and facilitate a closer coordination of their activities.
\end{enumerate}

\section{Grand Marshal}
The Grand Marshal shall be the presiding officer of the Student Senate and an ex-officio member of all committees.

\begin{enumerate}
\item The Grand Marshal shall be regarded as the leader and the chief spokesperson for the entire Union.
\item The Grand Marshal shall be a non-voting member of the Student Senate, but may cast one vote in the event of a tie.
\item Any appointments made by the Grand Marshal may be nullified by a majority vote of the total voting membership of the Student Senate.
\end{enumerate}

\section{President of the Union}
The President of the Union shall be the presiding officer of the Executive Board, and an ex-officio member of all Union committees.

\begin{enumerate}
\item The President of the Union shall be a non-voting member of the Executive Board but may cast one vote in the event of a tie.
\item Any appointments made by the President of the Union other than members of the Executive Board must be ratified by a $\frac{2}{3}$ vote of the Executive Board
\end{enumerate}

\section{Chairperson of the Judicial Board}
The Chairperson of the Judicial Board shall be the presiding officer of the Judicial Board.

\begin{enumerate}
\item The Chairperson shall be a non-voting member of the Judicial Board, but in the event of the tie may cast one vote.
\end{enumerate}

\section{President of the Graduate Council}
The President of the Graduate Council shall be the presiding officer of the Graduate Council and an ex-officio member of all graduate committees.

\begin{enumerate}
\item The President of the Graduate Council shall be elected by the Graduate Council from among its members.
\item His or her voting status will be determined via the by-laws of the Graduate Council. 
\end{enumerate}

\section{President of the Undergraduate Council}
The President of the Undergraduate Council shall be the presiding officer of the Undergraduate Council and an ex-officio member of all undergraduate committees.

\begin{enumerate}
\item The President of the Undergraduate Council shall be a non-voting member of the
Undergraduate Council but may cast one vote in the case of a tie.
\end{enumerate}

\section{Term Limits}
\begin{enumerate}
\item The offices of Grand Marshal and President of the Union are open to any student who has been
enrolled in Rensselaer for at least one full semester.
\item A student may hold the position of either Grand Marshal or President of the Union for no more than a combined total of two full terms.
\end{enumerate}

\article{DIRECTOR OF THE UNION}
The Director of the Union shall be employed by the Union to perform duties including the administration of Union activities and funds in accordance with the policies of the Rensselaer Board of Trustees, the Union Constitution, and further legislation of the Executive Board.

\vspace{12pt}

In maintaining advisory status, the Director may not veto any Executive Board action but may request the Student Senate to begin impeachment proceedings against any member of the Student Senate or the Executive Board whom he or she feels is fiscally irresponsible in the handling of Union funds.

\vspace{12pt}

Should the Director overstep his or her constitutional authority or demonstrate serious financial irresponsibility, a majority of the Executive Board may bring charges against the Director in the name of the Board before the Judicial Board.

\vspace{12pt}

The Vice President for Student Life shall act as adviser to the Director and shall inform him or her of Trustee policy.

\article{TREASURER OF THE UNION}
The Treasurer shall be the Treasurer of Rensselaer.

\vspace{12pt}

The Treasurer shall be the custodian of the funds of the Union and shall receive copies of all financial transactions.

\article{ORGANIZATIONS}
The Union shall be governed by a Student Senate, an Executive Board, a Judicial Board, a Graduate
Council, and an Undergraduate Council.
\begin{enumerate}
\item It shall be constituted under the authority of the Board of Trustees of Rensselaer.
\end{enumerate}


\section{Student Senate}

The Student Senate shall be the chief legislative and policy-making branch of the Union.

\begin{enumerate}
\item The Student Senate shall be presided over by the Grand Marshal.
\item The Student Senate shall consist of 26 voting members: Six Senators from the Graduate Class, four Senators from each undergraduate class, two Independent Senators, and two Greek Senators.
\item It shall be responsible for initiating all legislative action in those areas which affect the Union.
\item It shall pass such by-laws as are necessary for its operation by a $\frac{2}{3}$ vote of its total voting membership.
\item It shall determine the amount of the Union Activity Fee by a $\frac{2}{3}$ vote of its total voting membership.
\item It shall establish procedures for selection and removal of the student members of the committees of the Board of Trustees and all other appointments.
\item It shall be the responsibility of all senators to regularly report to their respective councils.
\end{enumerate}





\section{Executive Board}

The Executive Board shall act as the primary budgeting body of the Union.

\begin{enumerate}
\item The Executive Board shall be presided over by the President of the Union.
\item The Executive Board shall consist of a minimum of 15 voting members, and a maximum of 20
voting members:
\begin{enumerate}
\item The following positions on the Executive Board must be filled: 5 club or intercollegiate athletics representatives, 1 representative each from the Freshman, Sophomore, Junior, Senior, and Graduate classes, a representative of the Graduate Council, a representative of the Undergraduate Council, a Senate-Executive Board Liaison, and 2 Members-at-Large representing any membership of the Union.
\item The following positions on the Executive Board may be filled at the discretion of the President of the Union: 2 club or intercollegiate athletic members, and 3 Members-at-Large.
\item The Senate-Executive Board Liaison, Graduate Council Representative, and Undergraduate Council Representative are appointed by the Senate, Graduate Council, and Undergraduate Council respectively. All other positions are appointed by the President of the Union and approved by a $\frac{2}{3}$ vote of the Senate.
\item If a member of the Executive Board is unable to fulfill his or her duties, a replacement may be appointed by the President of the Union, subject to a majority vote of the Executive Board and a $\frac{2}{3}$ approval of the Senate.
\item The Director of the Union and his or her assistants shall act as advisers to the Executive Board.
\end{enumerate}


\item Such by-laws, as are necessary for its operation, shall be passed by a $\frac{2}{3}$ vote of its total voting
membership followed by a majority vote of the Student Senate.
\item It shall investigate, prepare, and approve the budget for the following fiscal year. The budget
and the records of all additional expenditures and appropriations shall be public documents. 
\item It shall administer the Union budget, appropriate Union funds and manage the business affairs
of any and all facilities operated by the Rensselaer Union.
\item It shall approve the hiring and continuance of all administrative personnel of the Union.
\item It shall decide on the classification of new organizations and changes in the classification of old
ones, as specified in its by-laws.
\end{enumerate}

\section{Judicial Board}
\begin{enumerate}
\item The Judicial Board shall be the primary court of judgment for cases in which the concept of student rights, responsibilities, or conduct is in question.
\begin{enumerate}
\item No other board shall assume any portion of this jurisdiction, unless such a board be comprised solely of students, and the jurisdiction of the board clearly defined.
\item No such board will be constituted nor its jurisdiction changed without the consent of the Student Senate and Institute President.
\end{enumerate}
\item The Judicial Board shall be the sole organization responsible for the interpretation of the Union Constitution.
\item The Judicial Board Chairperson shall be the presiding officer and a non-voting member of the
Judicial Board.
\item The Judicial Board shall consist of 6 voting members and 5 alternate members, all of whom must
be members of the Rensselaer Union.
\begin{enumerate}
\item The chairperson shall appoint new candidates to the Judicial Board subject to a $\frac{2}{3}$ approval of the Judicial Board followed by a $\frac{2}{3}$ approval of the Student Senate.
\item Alternate members are non-voting members of the Judicial Board who may assume voting rights under circumstances specified in the Judicial Board bylaws.
\item All appointments become effective when a vacancy exists in the position to which they are appointed. Appointments shall become effective in the order of their approval by the Judicial Board.
\item Once appointed, members of the Judicial Board remain active until they are no longer a member of the Rensselaer Union, have been removed from office, or have resigned their position.
\end{enumerate}
\item Any charge which specifies a violation of the Union Constitution, the Rensselaer Student Bill of
Rights, or established regulations governing student conduct may be referred to the Judicial
Board. In deciding a case, the Judicial Board may declare any legislation of the Student Senate,
the Executive Board, the Graduate Council, the Undergraduate Council, or the Undergraduate
Class Councils to be unconstitutional.
\item Such by-laws, as are necessary for its operation, shall be passed by a $\frac{2}{3}$ vote of its total voting
membership followed by a majority vote of the Student Senate.

\end{enumerate}

\section{Graduate Council}
\begin{enumerate}
\item The Graduate Council is the graduate legislative branch of the Union. Its voting membership shall consist of the six Graduate Senators and five representatives elected at large. It shall be presided over by the President of the Graduate Council.
\item It shall be responsible for initiating all legislative action within the Union pertaining to the graduate students not specifically dealt with by the Student Senate or the Executive Board.
\item It shall recommend any change in the amount of graduate Class Dues by a majority vote, subject
to a $\frac{2}{3}$ approval of the Student Senate’s total voting membership.
\item The Graduate Council shall be responsible for the expenditure of graduate Class Dues, collected from the members of its class by majority vote.
\item Such by-laws, as are necessary for its operation, shall be passed by a $\frac{2}{3}$ vote of its total voting membership followed by a majority vote of the Student Senate.
\end{enumerate}

\section{Undergraduate Council}
\begin{enumerate}
\item The Undergraduate Council is the undergraduate legislative branch of the Union. It shall consist of the Class President and Vice President of each Undergraduate Class as voting members, and shall be presided over by the President of the Undergraduate Council.
\begin{enumerate}
\item It shall be responsible for coordination and oversight of the four Undergraduate Class Councils.
\item All expenditures of Class Dues by the undergraduate Class Councils are subject to approval by a majority vote of the Undergraduate Council, in accordance with guidelines set forth in the by-laws of the Undergraduate Council.
\item It shall recommend any change in the amount of undergraduate Class Dues by a majority vote, subject to a $\frac{2}{3}$ approval of the Student Senate’s total voting membership.
\item Such by-laws, as are necessary for its operation, shall be passed by a $\frac{2}{3}$ vote of its total voting membership followed by a majority approval of the Student Senate.
\end{enumerate}
\item Each of the Senior, Junior, Sophomore, and Freshman classes shall elect a Class Council. Class Councils exist to promote the interests and unity of their Class.
\begin{enumerate}
\item The Class President is the presiding officer and a non-voting member of the Class
Council. The President may vote to break a tie in any vote of the Class Council.
\item The Class Vice President is an officer and voting member of the Class Council.
\item Voting membership in each Class Council shall include a minimum of eight voting Representative positions, their respective Class Senators, and the Vice President. Additional positions may be defined in the by-laws of the Undergraduate Council.
\item The President and Vice President of each Class Council shall serve as voting members on the Undergraduate Council.
\item Each Class Council shall recommend the expenditure of Class Dues, collected from the members of its class, by majority vote, subject to majority approval of the Undergraduate Council.
\item Each Class Council shall pass legislation commensurate with its duties, subject to regulations established in the bylaws of the Undergraduate Council and this Constitution.
\end{enumerate}
\item No person may hold more than one officer position, as defined by this Constitution or the Undergraduate Council Bylaws, on the same Class Council. The Undergraduate Council President may not hold a voting position nor be an officer of any Class Council.
\end{enumerate}

\section{Parliamentary Authority}
The rules contained in the current edition of Robert's Rules of Order shall govern the subordinate bodies of the Rensselaer Union, as defined in the Rensselaer Union Constitution, in all cases to which they are applicable and in which they are not inconsistent with the Rensselaer Union Constitution and any special rules of order the Union or its subordinate bodies may adopt.

\article{ELECTIONS}
\section{}
The general elections of the Grand Marshal, the President of the Union, the President of the Undergraduate Council, the elected members of the Sophomore, Junior and Senior Class Councils, the members of the Graduate Council, and all elected members of the Student Senate except those representing the incoming Freshman Class shall be held during the spring semester at a time and place determined by the Student Senate.

\section{}
\begin{enumerate}
\item The conclusion of an election is defined as the time at which its results are released.
\item The results of an election may be released no later than the end of the semester in which the
election is held.
\end{enumerate}

\section{}
\begin{enumerate}
\item The elections of the Freshman Class Council and Freshman members of the Student Senate shall
be held during the fall semester at a time and place determined by the Student Senate.
\item The Officers, Representatives, and Senators for the Freshman Class shall serve for a term
beginning on the date of their election and ending with the conclusion of the next general
election for their respective position.
\end{enumerate}

\section{}
Voting shall be by secret ballot. Only members of the Union at the time of the elections shall have the right to vote or run for office.

\section{Greek Status}
\begin{enumerate}
\item To be eligible for the office of Independent Senator, a student must not be a member of a
fraternity or sorority at Rensselaer.
\item Fraternal affiliation shall be determined by the Dean of Students Office.
\item To be eligible for the office of Greek Senator, a student must be a member of a fraternity or
sorority at Rensselaer. Of the candidates for Greek Senator, the fraternity member with the
highest number of votes and the sorority member with the highest number of votes will be
elected as Greek Senators.
\end{enumerate}

\section{}
\begin{enumerate}
\item The Student Senate shall determine the means whereby a student may have his or her name
placed on the ballot.
\item It shall also be responsible for providing adequate notice of all upcoming elections to the
members of the Union.
\item The Student Senate will be responsible for maintaining proportionate representation of all
members of the Union on the Student Senate, the Graduate Council, the Undergraduate
Council, the Executive Board, and the Judicial Board.
\end{enumerate}

\section{}
All elected offices, unless otherwise specified in this Constitution, shall be for a term beginning upon the date of their election and ending with the conclusion of the next general election for their respective
positions.

\section{}
The candidates for Senators, representatives to the Graduate Council, the Undergraduate Class Councils,
and President of the Undergraduate Council receiving the highest numbers of votes shall be elected.

\section{}
\begin{enumerate}
\item Vacancies in any elective office may be filled by appointment, subject to the provisions of the
constitution or by-laws of the body to which the appointment is made.
\item Such an appointment shall be valid for the remainder of the term of office.
\end{enumerate}

\section{}
\begin{enumerate}
\item Forty-percent of the popular vote shall be necessary for election in the case of Grand Marshal
and the President of the Union.
\item A plurality shall be necessary for election in the case of Undergraduate Class Council officers.
\item The number of representatives to each class council will be specified by its constitution.
\end{enumerate}

\section{}
If no victor can be declared in an election, another election will be held between the tied candidates at a
time and place determined by the Student Senate in its by-laws.

\section{}
\begin{enumerate}
\item The installation of all newly elected persons shall be within one week following the conclusion of
general elections.
\item The President of the Union shall appoint all required members of the Executive Board, excluding
the Graduate Council representative, the Undergraduate Council representative, the Freshman
class member, and the Senate-Executive Board Liaison, within three weeks after the conclusion
of general elections for the President of the Union.
\begin{enumerate}
\item The Undergraduate Council representative shall be appointed by the Undergraduate
Council within three weeks following the conclusion of general elections for the
Undergraduate Class Councils.
\item The Graduate Council representative shall be appointed by the Graduate Council within
three weeks following the conclusion of general elections for the Graduate Council.
\item The Student Senate representative shall be chosen by the Student Senate within three
weeks following the conclusion of general elections for the Student Senate.
\item The President of the Union shall appoint the Freshman class member within three
weeks after the fall elections.
\item The President of the Union shall appoint any discretionary members by the end of the
4
th week of the fall semester.
\item Term of office shall normally be one year but a member will continue to serve until
he/she has been replaced.
\end{enumerate}
\item The Graduate Council shall appoint its at-large members at the first regular meeting following
the conclusion of general elections for the Graduate Council. They shall be selected by the full
eleven member Graduate Council from all applicants. Applications must be solicited for at least
two weeks prior to all selection meetings.
\end{enumerate}

\section{}
To be eligible for election and to hold office, a student must satisfy the eligibility rules established by the
Institute and the additional regulations established by the Student Senate.

\article{LIMITATION OF OFFICE}
\section{}
A candidate for Grand Marshal or President of the Union may simultaneously run for a Senator or
Representative position in his or her Class Council or the Graduate Council.
\begin{enumerate}
\item He or she may not run for an officer position on that Council, nor may he or she hold more than one office should he or she be elected to both.
\end{enumerate}

\section{}
The Grand Marshal shall not be a member of the Executive Board, nor any committee thereof, nor shall the President of the Union be a member of the Student Senate, nor shall either be a member of a constituent council of the Student Senate, nor of any Union judiciary body.

\section{}
No person may simultaneously run for more than one position on the same council.

\section{}
No person may run simultaneously for the offices of Grand Marshal and President of the Union. 

\article{REMOVALS}
\section{}
\begin{enumerate}
\item The Grand Marshal, the President of the Union, or the President of the Undergraduate Council
may be removed from office by a petition requiring his or her resignation signed by at least 50
percent of his or her constituents or by a $\frac{2}{3}$ vote of his or her constituents voting in a valid
removal election.
\item A valid removal election is one in which at least 20 percent of his or her constituents vote.
\item A removal election may be called by a $\frac{2}{3}$ vote of the Student Senate or by a petition signed by
at least 20 percent of his or her constituents followed by a majority vote of the Student Senate.
\end{enumerate}

\section{}
\begin{enumerate}
\item The Student Senate may remove a voting member for good cause, as specified in the Student
Senate by-laws, by a $\frac{2}{3}$ vote of its total voting membership.
\item Elected senators may be recalled by their constituencies. A $\frac{2}{3}$ vote of his or her constituents in
a valid removal election would constitute a recall. Such a removal election may be brought
about by a petition signed by 20 percent of his or her constituents or by a majority vote of the
Student Senate. Senators filling seats normally appointed by a council, may be removed by a
$\frac{2}{3}$ vote of the council.
\item Should a vacancy remain unfilled by the body which made the appointment that body shall
forfeit the vote of the unoccupied position.
\end{enumerate}

\section{}
\begin{enumerate}
\item Any members of the Executive Board whom have been installed by the Student Senate may be
removed by a $\frac{2}{3}$ vote of the Student Senate, or by the simple request of the President of the
Union supported by a majority vote of the Student Senate.
\item The members of the Executive Board appointed by the Graduate Council, Undergraduate
Council, or the Student Senate may be removed by a $\frac{2}{3}$ vote of the appointing body, or by the
simple request of the President of the Union supported by a majority vote of the appointing
body.
\end{enumerate}

\section{}
A student member of the Judicial Board or Review Board may be recalled for good cause by $\frac{2}{3}$ vote of the Student Senate.

\section{}
\begin{enumerate}
\item Any member of the Graduate Council may be removed from office for good cause as specified in
their by-laws, with a $\frac{2}{3}$ vote of the Graduate Council. Any member of the Undergraduate Class
Councils may be removed from office for good cause as specified in the Undergraduate Council
by-laws, by a $\frac{2}{3}$ of their total voting membership of their Class Council.
\item The person whom the action concerns shall not have the right to vote in the removal decision.
An impeachment vote, requiring majority approval, may be called from the floor at any Council
meeting. If the impeachment vote is successful, a removal vote will be held at the following
meeting. 
\end{enumerate}

\article{SUCCESSION}

\section{}
Should the position of Grand Marshal become vacant, within two weeks the Student Senate shall
appoint a new Grand Marshal from within its membership. His or her appointment shall require the
approval of $\frac{2}{3}$ vote of the Student Senate. The Chairperson of the Judicial Board will preside over this
appointment. Until the new Grand Marshal is appointed, the Student Senate by-laws shall dictate who
will assume the remaining duties of the Grand Marshal.

\section{}
Should the position of President of the Union become vacant, within two weeks the Executive Board
shall appoint from within its membership a new President. His or her appointment shall be confirmed
by a $\frac{2}{3}$ vote of the Student Senate. The Executive Board by-laws shall dictate who will preside over the
Executive Board until the new President of the Union is confirmed.

\section{}
Should the position of Judicial Board Chairman become vacant, within two weeks the Judicial Board shall
nominate a new Judicial Board Chairman from among the membership of the Judicial Board. The
nominee shall be confirmed by a two-thirds vote of the Senate. Until a new Chairman has been
confirmed, the Judicial Board by-laws shall dictate who will assume the duties of the Chairman.

\section{}
Should the position of the President of the Undergraduate Council become vacant, all active Class
Councils shall meet in joint session within two weeks, for the purpose of electing a new President of the
Undergraduate Council. His or her appointment shall require a $\frac{2}{3}$ approval of the Student Senate. The
Grand Marshal shall preside over this appointment. Until the new President of the Undergraduate
Council is appointed, the Undergraduate Council By-laws shall dictate who will assume the remaining
duties of the President.

\section{}
If any elective position including by not limited to, Officers of the Union, is filled by appointment, a
special election for that position shall be held if the Student Senate receives, within two weeks of the
announcement of the appointment, a petition signed by 20 percent of the constituency involved. If the
appointment is made at a time when the full student body is not normally on campus, the petition must
be received within two weeks of the announcement to the full campus.

\section{}
Any vacant Class Senator position shall be filled by a person appointed by their corresponding Class
Council.

\article{PETITIONS AND REFERENDA}
\section{Referendum}
\begin{enumerate}
\item A referendum vote, on any matter except amendment of the Union Constitution, shall be held
whenever the Student Senate:
\begin{enumerate}
\item Approves any form of referendum question proposed by a Senator, Committee, Council,
or petition, by a $\frac{2}{3}$ vote.
\item Receives a petition signed by at least 10\% of all members of the Union, and does not
reject it. Such a petition may only be rejected by a $\frac{2}{3}$ vote of the total voting
membership of the Senate.
\end{enumerate}
\end{enumerate}

\section{}
The referendum election shall be held at a time determined by the Student Senate, not more than fortyfive
days (exclusive of vacation periods) or at the next regular election, whichever comes first, but in no
case less than fourteen days following its approval. A referendum may be held at any time during the
academic year except during exam periods.

\section{}
A referendum shall become effective if approved by a majority vote of students in a valid referendum
election. A valid referendum election is one in which at least 20 percent of the affected constituency
vote.

\section{}
Upon its passage, a referendum shall be binding upon the constituency. The Student Senate shall enact
any legislation necessary for its implementation. Unless otherwise provided for in the text of the
referendum, it may be rescinded only by a similar referendum. 

\article{AMENDMENTS}

\section{}
(a) Any amendment of the Union Constitution must be passed by a $\frac{2}{3}$ vote of the Student Senate
followed by a majority vote of the students voting in a valid constitutional election.
(b) A valid constitutional election shall be one in which 20 percent of those eligible vote.

\section{}
(a) One month prior to a general student body vote on the amendment, it must be submitted to the
President of the Institute or his/her chosen representative who may refer the amendment to 
the Board of Trustees for their consideration.

\section{}
(a) At least two weeks prior to the general student body vote on the amendment, the exact
wording of the amendment to be voted on shall be made available to the student body through
a campus-wide publication.

\section{}
(a) An amendment vote may be held at any time during the school year excluding exam periods.

\section{}
(a) A record of all amendments shall be appended to the Rensselaer Union Constitution for
posterity.

\article{LIABILITY}

No liability shall be incurred by this Union through its governing bodies, officers, or members unless
provision has been made for it. No liability shall be incurred or money spent by any committee, officer,
or member of the Union without a requisition from the Executive Board or its duly authorized agent. 

\end{document}